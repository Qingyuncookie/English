\documentclass[10pt,twocolumn,letterpaper]{article}

\usepackage{cvpr}
\usepackage{times}
\usepackage{epsfig}
\usepackage{graphicx}
\graphicspath{{/home/li/图片/}}
\usepackage{amsmath}
\usepackage{amssymb}
\usepackage{fontspec}

\usepackage[pagebackref=true,breaklinks=true,letterpaper=true,colorlinks,bookmarks=false]{hyperref}

\cvprfinalcopy % *** Uncomment this line for the final submission

\def\cvprPaperID{****} % *** Enter the CVPR Paper ID here
\def\httilde{\mbox{\tt\raisebox{-.5ex}{\symbol{126}}}}

\ifcvprfinal\pagestyle{empty}\fi
\setcounter{page}{1}

\begin{document}

\author{Qingyun Li\\\\
June 30, 2018}        
\title{Super Resolution}

\maketitle

\section{Tntroduction}
\par The  framework can yield specific training algorithms for many kinds of model and optimization algorithm. In the article, the authors explore the special case when the generative model generates samples by passing random noise through a multilayer perceptron, and this discriminative model is also a multilayer perceptron. We refer to this special case as adversarial nets. In this case, we can train both models and using only the highly successful backpropagation and dropout algorithms~\cite{Hinton2012Improving} and sample from the generative model using only forward propagation. No approximate inference or Markov chains are necessary.
  \bibliographystyle{ieee}
 \bibliography{single}
\end{document}
