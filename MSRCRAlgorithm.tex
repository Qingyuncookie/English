\documentclass{article}
\usepackage{graphicx}
\usepackage{subfigure}
\graphicspath{{/home/li/图片/}}
\usepackage{cite}
\usepackage{lscape}
\usepackage{multicol}
\author{Qingyun Li}
\date{May 11, 2018}
\title{MSRCR}
\begin{document}
\maketitle
 \par The Retinex theory was introduced by Land and McCann in 1971 and is based on the assumption of a Mondrian world. They argue that human color sensation appears to be independent of the amount of light, that is the measured intensity, coming from observed surfaces. 
 \par The MSRCR algorithm is an extension of the retinex algorithm\cite{Rahman1996Multiscale}. It achieves a balance with respect to dynamic range, edge enhancement, and color constancy, and can adaptively enhance various types of images, which makes it suitable for a wide range of applications, The retinex theory is the basis of an image enhancement method proposed by Jobson \emph {et al} \cite{Lim2014Image}. This method uses the theory of color constancy, and is a model that can be solved by mathematical calculation.
\bibliographystyle{plain}
\bibliography{single}
\end{document}