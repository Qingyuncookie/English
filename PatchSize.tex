\documentclass{article}
\usepackage{graphicx}
\usepackage{subfigure}
\graphicspath{{/home/li/图片/}}
\usepackage{multicol}
\usepackage{lscape}
\author{Qingyun Li}
\date{April 27, 2018}
\title{Patch Size}
\begin{document}
\maketitle
 \par It's fattal of the estimation of transmission map for the image dehazing, the equation as you see at (1).
 \begin{equation}
\tilde{t}(x)=1-\min\limits_{y\in\Omega(x)}(\min\limits_{c}\frac{I^{c}(y)}{A^c})
\end{equation}
\par And in fact, another key parameter in the algorithm is the patch size in (1). On one hand, the dark channel prior becomes better for a larger patch size because the probability that a patch contains a dark pixel is increased. We can see at the Fig.\ref{patch size}: the larger the patch size, the darker the dark channel.
\begin{figure}[htbp]
\begin{minipage}{0.3\linewidth}
\includegraphics[width=1\linewidth]{a.png}\\
\label{fig:side:a} 
\end{minipage}
\hfill
\begin{minipage}{0.3\linewidth}
\includegraphics[width=1\linewidth]{b.png}\\
\label{fig:side:b}
\end{minipage} 
\hfill
\begin{minipage}{0.3\linewidth}
\includegraphics[width=1\linewidth]{c.png}\\
\label{fig:side:c} 
\end{minipage}
\caption{A haze-free image and its dark channels using 3×3 and 15×15 patches, respectively.} \label{patch size}
\end{figure}
\end{document}