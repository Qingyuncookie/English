\documentclass{article}
\usepackage{picinpar,graphicx}
\usepackage{multicol}
\usepackage{lscape}
\author{Qingyun Li}
\date{May 2, 2018}
\title{Atmosphric Light}
\begin{document}
\maketitle
 \par According to the DCP\cite{He2011Single} approximation of $J^{dark}$ $\approx$0, the transmssion map $\tilde{t}(x)$ can be represented as
\begin{equation}
\tilde{t}(x)=1-\min\limits_{y\in\Omega(x)}(\min\limits_{c}\frac{I^{c}(y)}{A^c})
\end{equation}
\par Here, the atmosphric A needs to be estimated. And the Table 1 lists the conventional methods that are uesd to estimate atmosphric light. 
\begin{table}[htbp]
\centering
\caption{Conventional methods used to estimate A}
\begin{tabular}{|c|c|c|}
\hline
Parameter & Selection criterion & Reference \\
\hline 
p=0 & Highest intensity & \cite{Long2013Fast} \\
\hline 
p=0.1 & Highest intensity & \cite{He2011Single} \\
\hline 
p=0.2 & Highest intensity & \cite{Xiao2012Gan} \\
\hline 
p=0.1 & Minimum entropy & \cite{Jeong2013The} \\
\hline
\end{tabular} 
\end{table}
\bibliographystyle{plain}
\bibliography{single}
\end{document}