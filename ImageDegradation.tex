\documentclass{article}
\usepackage{picinpar,graphicx}
\graphicspath{{/home/li/图片/}}
\usepackage{multicol}
\usepackage{lscape}
\usepackage{cite}
\author{Qingyun Li}
\date{May 23, 2018}
\title{Image Degradation}
\begin{document}
\maketitle
 \par Last time, we talked about the differece of fog and smog, today, we will learn how does the degradation image developed. 
 \par The atmosphric light and the light reflected by the object itself combine with each other. And then, they enter the image acquisition device to form image. As is shown at Fig.~\ref{model1} So the suspended particles in the air and the incident light all effect the scattering effect of the atmosphere. When fog weather occurs, the amount of suspended particles increases. Therefore, in the process of image collection, the scattering and absorption of the radiated light is strengthened, and this resulted the color degradation and blurred details in the image.
 \begin{figure}[htbp]
\begin{minipage}{1\linewidth}
\centering{}
\includegraphics[width=0.7\linewidth]{model1.jpeg}\\
\caption{Atmospheric optical imaging model}\label{model1}
\end{minipage}
\end{figure}
\bibliographystyle{plain}
\bibliography{single}
\end{document}