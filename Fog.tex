\documentclass{article}
\usepackage{picinpar,graphicx}
\usepackage{multicol}
\usepackage{lscape}
\usepackage{cite}
\author{Qingyun Li}
\date{May 21, 2018}
\title{Fog}
\begin{document}
\maketitle
 \par As we all know, fog is a natural weather phenomenon. Especially recent yeays, due to the distruction of the   environment, this phenomenon is more and more serious. The buildings is higher and the air isn't free, all this reasons lead to the formation of the fog. And our common life has been effected seriously. 
 \par Today we discuss the concentration level of fog, we can learn it explicitly at the table 1.
\begin{table}[htbp]
\centering
\caption{Concentration level of fog}
\begin{tabular}{c|c}
\hline
Level & Horizontal Visibility Distence \\
\hline
mist & 1000-10000 \\
fog & 500-1000 \\
heavy fog & 200-500 \\
smoke fog & 50-200 \\
strong fog & <50 \\
\hline 
\end{tabular} 
\end{table}
\end{document}