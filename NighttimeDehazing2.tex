\documentclass{article}
\usepackage{multicol}
\usepackage[top=1in,bottom=1in,left=1.25in,right=1.25in]{geometry}
\usepackage{lscape}
\author{Qingyun Li}
\date{April 15, 2018}
\title{Nighttime Haze Removal}
\begin{document}
\maketitle
\par Last time, we learned that nighttime dehazing is different from the daytime dehazing, so the authors introduce a new model. Today, we will learn it detailedly.
\par First, we describe the standard daytime dehazing model and it's also the most commonly used: I(x) = R(x)t(x) + L(1 − t(x)), where I(x) is the observed picture, t(x) is the transmission that indicates the portion of the particles, and R(x) represents the scene reflection or rediance when there is no haze. and L is assumed to be globally uniform. Inspired by this, we modified standard haze model by adding the glow model into it. The new model is I(x) = R(x)t(x)+L(x)(1−t(x))+L$ _a $(x)∗APSF, where L$ _a $ is the active light sources, which the intensity is convolved with the atmosphere point spread function, APSF. And it can change at different locations, this because various colors from different light sources can contribute to the atmosphere light as a result of the scattering process. It offers a useful means to describe nighttime haze images with glow and active light sources.
\end{document}