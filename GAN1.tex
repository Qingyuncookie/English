\documentclass[10pt,twocolumn,letterpaper]{article}

\usepackage{cvpr}
\usepackage{times}
\usepackage{epsfig}
\usepackage{graphicx}
\graphicspath{{/home/li/图片/}}
\usepackage{amsmath}
\usepackage{amssymb}
\usepackage{fontspec}

\usepackage[pagebackref=true,breaklinks=true,letterpaper=true,colorlinks,bookmarks=false]{hyperref}

\cvprfinalcopy % *** Uncomment this line for the final submission

\def\cvprPaperID{****} % *** Enter the CVPR Paper ID here
\def\httilde{\mbox{\tt\raisebox{-.5ex}{\symbol{126}}}}

\ifcvprfinal\pagestyle{empty}\fi
\setcounter{page}{1}

\begin{document}

\author{Qingyun Li\\\\
July 20, 2018}        
\title{GAN}

\maketitle

\section{Introduction}
\section{Related work}
\section{Adversarial nets}
\section{Theoretical results}
\par The generator G implicitly defines a probability distribution $p_{g}$ as the distribution of the samples G(z) obtained when z$\sim$p(z). Therefore, we should like Algorithm 1 to converge a good estimator of $p_{data}$, if given enough capacity and training time. The result of this section are done in a nonparametric setting, e.g.\ we represent a model with infinite capacity by studying convergence in the space of probability density functions.
\subsection{Global optimality of $p_{g}=p_{data}$}
\par At first the authors consider the optimal discriminator D for any given generator G.
\par $\mathbf{Proposition 1.}$ For G fixed, the optimal discriminator D is
\begin{equation}
D_{G}^{*}(x)=\frac{p_{data}(x)}{p_{data}(x)+p_{g}(x)}
\end{equation}
\par The training criterion for the discriminator D, given any generator D, given any generator G, is to maximize the quantity V(G,D)
\begin{equation}
\begin{split}
V(G,D)=\int_{x}p_{data}(x)log(D(x))dx\\
+\int_{z}p_{z}(z)log(1-D(g(z)))dz\\
=\int_{x}p_{data}(x)log(D(x))\\
+p_{g}(x)log(1-D(x))dx
\end{split}
\end{equation}
  \bibliographystyle{ieee}
 \bibliography{single}
\end{document}
