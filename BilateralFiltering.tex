\documentclass{article}
\usepackage{graphicx}
\usepackage{subfigure}
\graphicspath{{/home/li/图片/}}
\usepackage{cite}
\usepackage{lscape}
\usepackage{multicol}
\author{Qingyun Li}
\date{May 7, 2018}
\title{Bilateral Filtering}
\begin{document}
\maketitle
 \par A bilateral filter is a non-linear, edge-preserving, and noise-reducing smoothing filter for images. It replaces the intensity of each pixel with a weighted average of intensity values from nearby pixels. This weight can be based on a Gaussian distribution. Crucially, the weights depend not only on Euclidean distance of pixels, but also on the radiometric differences. This preserves sharp edges, as is shown at Fig.~\ref{image}.
 \par In mathematics, the Euclidean distance or Euclidean metric is the "ordinary" straight-line distance between two points in Euclidean space. And the Euclidean distance between points p and q is the length of the line segment connecting them.
\begin{figure}[htbp]
\begin{minipage}{1\linewidth}
\centering
\includegraphics[width=0.7\linewidth]{Bilateral_Filter.png}\\
\caption{Left: original image. Right: image processed with bilateral filter}\label{image} 
\end{minipage}
\end{figure}
\end{document}