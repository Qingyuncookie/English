\documentclass{article}
\usepackage{graphicx}
\graphicspath{{/home/li/图片/}}
\usepackage{multicol}
\usepackage{lscape}
\author{Qingyun Li}
\date{April 21,2018}
\title{Transmission Map}
\begin{document}
\maketitle
\par Last time, we talked about DCP. Due to this doscovery, the transmssion map$\tilde{t}(x) $is obtained from the DCP accordding to the equation:
\begin{equation}
\tilde{t}(x)=1-\min\limits_{y\in\Omega(x)}(\min\limits_{c}\frac{I^{c}(y)}{A^c})
\end{equation}
\par However, in fact, the pixel value of dark channel, $J^{dark}(x)$, is not equivalent to zero completely. So, in order to make the image looked natural, we need to retain a small amount of haze by using a constant $\omega$ (0<$\omega$<1):\begin{equation}
\tilde{t}(x)=1-\omega \min\limits_{y\in\Omega(x)}(\min\limits_{c}\frac{I^{c}(y)}{A^c})
\end{equation}
And inadvertently, we compensate for the under-estimation of $\tilde{t}(x)$ by multiplying $\omega$. 
\begin{figure}[htbp]
\begin{minipage}{0.5\linewidth}
\centering{}
\includegraphics[width=0.9\linewidth]{image1.png}\\
\caption{haze image}\label{image} 
\end{minipage}
\hfill
\begin{minipage}{0.5\linewidth}
\centering{}
\includegraphics[width=0.9\linewidth]{map1.png}\\
\caption{transmission map}\label{map}
\end{minipage} 
\end{figure}
\end{document}