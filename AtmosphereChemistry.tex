\documentclass{article}
\usepackage{picinpar,graphicx}
\usepackage{multicol}
\usepackage{lscape}
\author{Qingyun Li}
\date{May 13, 2018}
\title{Atmosphere Chemistry}
\begin{document}
\maketitle
 \par 
Mean atmospheric water vapor
The three major constituents of Earth's atmosphere, are nitrogen, oxygen, and argon. Water vapor accounts for roughly 0.25$\%$ of the atmosphere by mass. The concentration of water vapor varies significantly from around 10 ppm by volume in the coldest portions of the atmosphere to as much as 5$\%$  by volume in hot, humid air masses, and concentrations of other atmospheric gases are typically quoted in terms of dry air (without water vapor).As is shown at table 1. 
\begin{table}[htbp]
\centering
\caption{Major constituents of dry air, by volume}
\begin{tabular}{|c|c|c|c|}
\hline
\multicolumn{2}{|c|}{Gas} & \multicolumn{2}{|c|}{Volume} \\
\hline 
Name & Formular & in $ppmv^{B}$ & in $\%$ \\
\hline 
Nitrogen & $N_{2}$ & 780840 & 78.084 \\
\hline 
Oxygen & $O_{2}$ & 209460 & 20.946 \\
\hline 
Argon & Ar & 9340 & 0.9340 \\
\hline
Carbon dioxide & $CO_{2}$ & 400 & 0.04 \\
\hline
Neon & Ne & 18.18 & 0.001818 \\
\hline
\end{tabular} 
\end{table}
\bibliographystyle{plain}
\bibliography{single}
\end{document}