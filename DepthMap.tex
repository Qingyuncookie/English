\documentclass{article}
\usepackage{graphicx}
\usepackage{subfigure}
\graphicspath{{/home/li/图片/}}
\usepackage{cite}
\usepackage{lscape}
\usepackage{multicol}
\author{Qingyun Li}
\date{April 29, 2018}
\title{Depth Map}
\begin{document}
\maketitle
 \par Last time, we talked about atmosphric light. To quantitativly evaluate A estimation methods, we used the foggy rode image database (FRIDA) \cite{Tripathi2012Removal} which consists of pairs of synthetic color and depth map such as Fig.~\ref{depth map}. In Wiki, the depth map is explained as "In 3D computer graphics a depth map is an image or image channel that contains information relating to the distance of the surfaces of scene objects from a viewpoint". 
\begin{figure}[htbp]
\begin{minipage}{0.5\linewidth}
\includegraphics[width=0.9\linewidth]{image3.png}\\
\caption{Cubic Structure}\label{original image} 
\end{minipage}
\hfill
\begin{minipage}{0.5\linewidth}
\includegraphics[width=0.9\linewidth]{image3depth.png}\\
\caption{Depth Map}\label{depth map}
\end{minipage} 
\end{figure}
\bibliographystyle{plain}
\bibliography{single}
\end{document}